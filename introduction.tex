
\section{Introduction}
\label{sec:introduction}

%In this work, we consider the problem of graph clustering where communities of highly related nodes
%are identified based on their interconnections. This has significant applications in studying social
%and biological systems where related entities often interact more with each other and therefore form
%dense subgraphs where an edge represent an interaction. Although the notion of community is
%intuitive, the challenge, however, is to find a precise mathematical definition that efficiently
%identifies the communities of interest.

In this work, we consider the problem of graph clustering, see e.g., \cite{fortunato2016community}, where communities of highly related nodes
are identified based on their interconnections. This has significant applications in studying social
and biological systems where related entities tend to interact more with each other compared to
unrelated ones.
Although the notion of community is
intuitive, the challenge, however, is to find a precise mathematical definition that efficiently
identifies the communities of interest.

The most widely accepted definition of a community, or graph cluster, is perhaps that of
\cite{flake:efficient,radicchi:defining,newman2004fast}. Roughly speaking, a community
is a group of nodes  that share
stronger connections among themselves compared with the rest of the graph. Such a definition naturally inspires
 graph cut \cite{flake:cut-clustering,shi2000normalized},
modularity \cite{newman2004fast,blondel2008fast}, random walk \cite{rosvall2008maps} based
methods, to name a few, for identifying communities. These methods have to refine the original
definition of a community for efficient computation and storage. In particular, web communities
defined in \cite{flake:efficient,flake:cut-clustering} are NP-hard to compute, and so a
cut-clustering algorithm was proposed in \cite{flake:cut-clustering} that finds approximate web
communities by minimizing a weighted sum of the external connections and the size of the community.
It was shown that the computation reduces to finding the mincuts of some modified graphs,  and the
set of all such communities, or cut-clusters, form a hierarchy, which can be represented efficiently
by a dendrogram.

However, the hierarchical structure of cut-clusters rely on the graph being undirected. Furthermore,
we find that cut-clustering often fails to return dense subgraphs but instead returns the less
interconnected peripherals of the dense subgraphs. This behaviour is not surprising since, for a given
size, a set of nodes having fewer external connections does not necessarily imply they have more
internal connections. In other words, a dense subgraph may have many external connections
with the rest of the graph compared with the external connections of a sparser subgraph.
In this work, we primarily focus on improving the cut-clustering algorithm. Our proposed solution
not only resolves the problem with no additional cost in computation and storage, but also
extends to cluster digraphs or more general non-graphical networks with a submodular cost
function. 

This paper is organized as follows. In Section~\ref{sec:preliminaries}, we introduce the notion of
web communities for digraphs and the approximate solution by cut-clustering for undirected graphs.
In Section~\ref{sec:problem}, we improve the formulation of cut-clustering and extend it to
digraphs. In  Sections~\ref{sec:hierarchy} and \ref{sec:computation}, we explain the hierarchical
structure of the communities and its polynomial-time computation. In Section~\ref{sec:experiment},
we give test results comparing the proposed algorithm to cut-clustering on two small example
graphs.
\ifPAGELIMIT
Detailed comparisons with cut-clustering and other algorithms can be found in \cite[Appendices~A and B]{long}, which include more experiments on both synthetic and real-world data sets. The proofs and detailed calculations of some of the examples are given in \cite[Appendices~C and D]{long}.
\else Detailed comparisons with cut-clustering and other algorithms can be found in Appendices~\ref{sec:comparison-cut} and \ref{sec:comparison-other} along with experiments on both synthetic and real-world data sets. The proofs and detailed calculations of some of the examples are given in Appendices~\ref{sec:proofs} and \ref{sec:calculations}.
\fi