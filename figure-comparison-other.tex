\tikzstyle{cluster}=[fill opacity=0.2,rounded corners,inner sep=0.1em]

\begin{figure}[t]
  \centering
  \tikzstyle{nodestyle}=[inner sep=0.2em,outer sep=0em]
  \tikzstyle{every to}=[every node/.style={font=\scriptsize}]
  \begin{subfigure}[b]{.45\textwidth}
    \centering
    \begin{tikzpicture}
      \matrix (DSP) [matrix of math nodes,column sep=1.2em, row sep=1em, nodes={nodestyle}]
      {
	|(0)| 0 & |(1)| 1 & |(2)| 2 & |(3)| 3 & |(4)| 4 \\
      };
      \draw (0) to node [above] {1} (1) (2) to node [above] {2} (3) to node [above] {1} (4);
      \node[cluster,fill=blue,fit=(0) (1)] {};
      \node[cluster,draw=red,fit=(2) (3)] {};
    \end{tikzpicture}
    \caption{Densest subgraphs: The set $\{0,1\}$ is a web-community that is not a densest $2$-subgraph since $\{2,3\}$ has strictly more internal support.}
    \label{fig:DSP}    
  \end{subfigure}
  %%%%%%%%%%%%%%%
  \hfill
  %%%%%%%%%%%%%%%
  \begin{subfigure}[b]{0.45\textwidth}
    % for drawing ring of clique
    % https://tex.stackexchange.com/questions/223854/drawing-a-ring-of-cliques-in-tikz-graphs/223902#223902
    \newcommand\single[2]{ % #1=labels, #2= n=number of nodes
      \foreach \x in {1,...,#2}{
        \pgfmathsetmacro{\ang}{360/#2}
        \pgfmathparse{(\x-1)*\ang}
        \node[draw,fill=red,circle,inner sep=1pt] (#1-\x) at (\pgfmathresult:4cm) {};
      }
      \foreach \x [count=\xi from 1] in {1,...,#2}{
        \foreach \y in {\x,...,#2}{
          \path (#1-\xi) edge[-] (#1-\y);
        }
      }
    }
    \centering
    \begin{tikzpicture}
    \begin{scope}[local bounding box=scope1]
    %\node at (0,0){$n=5, m=30$};
    \node at (0,0){$30$ cliques};
    \end{scope}
    \foreach \s[count=\si from 0] in {0,45,90,...,360}{
    \begin{scope}[shift={($(scope1) +(\s:2)$)}, scale=0.1,rotate=\s+90]
    \single{\si}{5};
    \end{scope}
    }
    \foreach \i/\j in {1/2,2/3,3/4,4/5,5/6,6/7,7/8}
    \draw (\i-1)--(\j-3);
    %\draw[dotted,thick, outer sep=10pt] (8-1)--(1-3);
	 \def\dd{.13cm}
    \draw[dotted,thick, outer sep=10pt] ([xshift=-\dd, yshift=\dd]8-1.north)--([xshift=\dd, yshift=-\dd]1-3.south);
    \end{tikzpicture}
	 %
	 \hfill
    \begin{tikzpicture}
      \matrix (DSP) [matrix of math nodes,column sep=1.9em, row sep=2em, nodes={nodestyle}]
      {
		|(0)| 0 \\ |(1)| 1 \\ |(2)| 2 \\ |(3)| 3 \\
      };
		\draw (0) to node [left] {1} (1) (1) to node [left, rotate=0] {$0.6$} (2) to node [left] {1} (3);
      \node[cluster,inner sep=.3em,fill=blue,fit=(0) (1)] {};
      \node[cluster,inner sep=.3em,fill=blue,fit=(2) (3)] {};
      \node[cluster,inner sep=.9em, draw=red,fit=(0) (3)] {};
    \end{tikzpicture}
    \caption{Modularity: Two graphs showing modularity's bias towards solutions of larger size.}
    \label{fig:modularity_2}
  \end{subfigure}
  %%%%%%%%%%%%%%%
  \hfill
  %%%%%%%%%%%%%%%
  \begin{subfigure}[b]{.45\textwidth}
    \centering
    \begin{tikzpicture}
      \matrix (con) [matrix of math nodes,column sep=1.2em, row sep=1em, nodes={nodestyle}]
      {
		|(0)| 0 & |(1)| 1 & |(2)| 2 & |(3)| 3 & |(4)| 4 & |(5)| 5 & |(6)| 6 & |(7)| 7 & |(8)| 8 & |(9)| 9\\
      };
      \draw (0) to node [above] {2} (1) to node [above] {0.6} (2) to node [above] {1} (3) to node [above] {1} (4) to node [above] {1} (5) to node [above] {1} (6) to node [above] {1} (7) to node [above] {1} (8) to node [above] {1} (9);
      \node[cluster,fill=blue,fit=(0) (1)] {};
      \node[cluster,draw=red,fit=(0) (1) (2) (3) (4),inner sep=0.4em] {};
      %\node[cluster,draw=red,fit=(5) (6) (7) (8) (9),inner sep=0.4em] {};
    \end{tikzpicture}
    \caption{Minimum conductance: The sets $\{0,1,2,3,4\}$ and $\{5,6,7,8,9\}$ minimize the conductance to $\frac19$. In contrast,
our approach can return the desired cluster
$\{0,1\}$.}
    \label{fig:conductance}    
  \end{subfigure}
  %%%%%%%%%%%%%%%
  \hfill
  %%%%%%%%%%%%%%%
  \begin{minipage}[b]{0.45\textwidth}
  \begin{subfigure}[b]{\textwidth}
    \centering
    \begin{tikzpicture}
      \matrix (con) [matrix of math nodes,column sep=1.2em, row sep=1em, nodes={nodestyle}]
      {
		|(0)| 0 & |(1)| 1 & |(2)| 2 & |(3)| 3 & |(4)| 4 & |(5)| 5 & |(6)| 6 & |(7)| 7 & |(8)| 8 & |(9)| 9\\
      };
      \draw (0) to node [above] {2} (1) to node [above] {0.6} (2) to node [above] {1} (3) to node [above] {1} (4) to node [above] {1} (5) to node [right] {1} (6) to node [above] {1} (7) to node [above] {1} (8) to node [above] {1} (9);
      \node[cluster,fill=blue,fit=(2) (3) (4) (5) (6) (7) (8) (9)] {};
      \node[cluster,draw=red,fit=(2) (3) (4) (5),inner sep=0.4em] {};
      \node[cluster,draw=red,fit=(6) (7) (8) (9),inner sep=0.4em] {};
      %\node[cluster,draw=red,fit=(5) (6) (7) (8) (9),inner sep=0.4em] {};
    \end{tikzpicture}
    \caption{Infomap: The sets $\{0,1\}$, $\{2,3,4,5\}$ and $\{6,7,8,9\}$ optimize the map equation~\cite{rosvall2008maps}. In contrast,
our approach can return the desired cluster $\{2,3,4,5,6,7,8,9\}$.}
    \label{fig:infomap}    
  \end{subfigure}\\
  \begin{subfigure}[b]{\textwidth}
    \centering
    \begin{tikzpicture}
      \matrix (con) [matrix of math nodes,column sep=1.2em, row sep=1em, nodes={nodestyle}]
      {
	|(0)| 0 & |(1)| 1 & |(2)| 2 & |(3)| 3 & |(4)| 4 & |(5)| 5\\
      };
      \draw (0) to node [above] {1} (1) to node [above] {2} (2) to node [above] {1} (3) to node [above] {0.5} (4) to node [above] {1} (5);
      % \node[cluster,fill=blue,fit=(0) (1)] {};
      \node[cluster,draw=red,fit=(0) (1) (2) (3),inner sep=0.4em] {};
      \node[cluster,draw=red,fit=(4) (5),inner sep=0.4em] {};
      %\node[cluster,draw=red,fit=(5) (6) (7) (8) (9),inner sep=0.4em] {};
    \end{tikzpicture}
    \caption{Infomap: will return \{\{0, 1, 2, 3\},\{4, 5\}\} but not the stronger community
	 $\{1,2\}$. In contrast, our formulation returns all of them parameterized by their strengths.}
    \label{fig:infomap2}
  \end{subfigure}
\end{minipage}
  %%%%%%%%%%%%%%%
  \hfill
  %%%%%%%%%%%%%%%
  \vspace{-.5em}
  \caption{Examples of weighted graphs where our formulation (highlighted in blue) can return more meaningful communities
  than existing methods (circulated in red). (Not all communities are highlighted / listed.)}
  \label{fig:eg_for_graphs}
\end{figure}
