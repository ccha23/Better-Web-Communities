
\section{Computation}
\label{sec:computation}

In this section, we will derive a polynomial-time algorithm to returned the proposed community hierarchy $\mcC$ for any given $`b\in [0,1]$. The algorithm is available on GitHub.\footnote{\url{https://github.com/handasontam/Alpha-Beta-Communities}}

\subsection{Divide-and-conquer}

%For a clearer discussion in the next two sections,
%we introduce the notion of community candidates in this section. Towards that end,
We first rewrite \eqref{eq:alpha-beta-community} as a two-step minimization
\begin{subequations}
	\begin{align}
		\label{eq:g-community}
		\hat{f}(`a) = &\min_{t\in V} \hat{f}_{t}(`a), \text{ where } \\
		\label{eq:g-candidate}
		\hat{f}_{t}(`a) := & \min_{C\subseteq V:t\in C} f_{`a}(C)
	\end{align}
        and $f_{`a}$ is defined in \eqref{eq:fab}.
\end{subequations}
The minimization in \eqref{eq:g-candidate} has a unique inclusion-wise
minimum solution due to a well-known result in submodular function
minimization (see~\cite{fujishige05}), and the fact that $f_{}$ (and thereby $f_{\alpha}$)
%is submodular~\eqref{eq:submodular}, i.e.,
is submodular, i.e., $\forall C_1,C_2\subseteq V$,
\begin{align}
  f_{}(C_1)+f_{}(C_2) \geq f_{}(C_1\cup C_2) + f_{}(C_1\cap C_2). %, \forall C_1,C_2\subseteq V.
  \label{eq:submodular}
\end{align}
This can be seen by rewriting $f_{}$, defined in \eqref{eq:fb}, using the identity $w(V`/C,C)+w(C,C)=\sum_{i\in C} d_i$ as
\begin{align}
  f_{}(C) = w(V`/C,C)-`b\sum_{i\in C} d_i,
\end{align}
which is submodular in $C$ since the graph cut $w(V`/C,C)$ and the sum $\sum_{i\in C} d_i$ are
modular~\cite{fujishige05}.   

\begin{definition}
	\label{def:candidates}
	For $`a \geq 0$, $`b\in [0,1]$, and $t\in V$, a community candidate, or
	candidate for short, $C_{`a,t}$ is defined as the inclusion-wise minimum solution to the minimization
	\eqref{eq:g-candidate}.
\end{definition}
(Similar to communities, we make the dependency on $\beta$ implicit for notational simplicity.)
For a given $\beta \in [0,1]$, let $T^*_{`a}$ be the set of optimal solutions $t$ to the
minimization \eqref{eq:g-community}, then we have the following proposition.
%Note that the set of $(`a,`b)$-communities is the set of candidates $C_{`a`bt}$ such
%that $|C_{`a`bt}| > 1$ and $t$ is an optimal solution to the outer minimization
%\eqref{eq:g-community}.
%\begin{proposition}
%	\label{prop:communities-from-candidates}
%	The set of inclusion-wise minimal solutions to \eqref{eq:alpha-beta-community} is given as 
%	\begin{align}
%		\op{minimal} \{C_{`a`bt} \mid |C_{`a`bt}| \geq 1, t\in T^*_{`a`b}\}, \label{eq:abt->ab}
%	\end{align}
%	where $\op{minimal}(\mcF)$ denotes the set of minimal elements in $\mcF$. Hence, the set of
%	communities $\mcS_{`a}$ can be obtained by removing the singleton elements from the
%	above.\footnote{Note again that \eqref{eq:abt->ab} with strict inequality $|C_{`a`bt}|> 1$
%	instead can be a proper superset of
%	$\mcC_{`a`b}$. 
%    }
%\end{proposition}
\begin{proposition}
	\label{prop:communities-from-candidates}
	The set of inclusion-wise minimal solutions to \eqref{eq:alpha-beta-community} is equal to the
	set of minimal candidates $C_{\alpha, t}$ s.t. $t\in T^{*}_{\alpha}$.
	%Hence, the set of communities $\mcS_{`a}$ can be obtained from the above by removing the
	%singleton elements, i.e., it is given as
	In other words, the set of communities $\mcS_{`a}$ can be written as 
	\begin{align}
		%\op{minimal} \{C_{`a`bt} \mid |C_{`a`bt}| \geq 1, t\in T^*_{`a`b}\}, \label{eq:abt->ab}
		\op{minimal} \{C_{`a, t} \mid t\in T^*_{`a}\} \ \backslash \ \{\Set{i} \mid i \in V\}, \label{eq:abt->ab}
	\end{align}
	%where $\op{minimal}(\mcF)$ denotes the set of minimal elements in $\mcF$.
\end{proposition}
%		\textcolor{magenta}{
%			aaa--- By using $T^{*}_{\alpha^{\text{+}}\beta}$ instead of $T^{*}_{\alpha\beta}$, i.e.,
%			\begin{align*}
%				%\op{minimal} \{C_{`a`bt} \mid |C_{`a`bt}| \geq 1, t\in T^*_{`a`b}\}, \label{eq:abt->ab}
%				\op{minimal} \{C_{`a`bt} \mid t\in T^*_{`a^{\text{+}}`b}\} \ \backslash \ \{\Set{i} \mid i \in V\},
%				%\label{eq:abt->ab}
%			\end{align*}
%			where
%		$\alpha^{\text{+}}-\alpha$ is an arbitrarily small positive number, the operator ``$\op{minimal}$'' can
%		be kept, dropped, or reblaced with``$\op{maximal}$'' without affecting the set. Note that two candidates with
%		memebers in $T^{*}_{\alpha^{\text{+}}\beta}$ must be equal or disjoint.  
%		}
    
\begin{proof}
	Follows directly from Definitions~\ref{def:alpha-beta} and \ref{def:candidates}.
\end{proof}

%%%%%%%%%%%%
%The following example demonstrates the proposition and the above definitions.
\begin{example}
	\label{eg:chain-community}
	Consider the undirected graph in Fig.~\ref{fig:eg-chain} with $V:=\Set{0,1,2,3}$.
%
	It is straightforward to enumerate all the web communities \eqref{eq:support-community} as
	\begin{align}
		\label{eq:chain-communities}
		\Set{0,1,2,3},
		\Set{0,1,2},
		\Set{1,2,3}, %\text{and}
		\Set{1,2}.
	\end{align}
%
	For $\beta = 0$, Fig.~\ref{fig:eg-chain-beta0}
	shows the plots (against $`a$) of the functions $\hat{f}(`a)$ (blue), $\hat{f}_{t}(`a)$
	(solid), and
	$f_{\alpha}(C)$ (dashed black).%
	\footnote{
	The figure shows $f_{\alpha}(C)$ only for the relevant
	subsets $C\subseteq V$ achieving \eqref{eq:g-candidate}, i.e., the candidates. In
	other words, all suppressed lines lie above the curve of $\hat{f}_{t}$ for all $t\in V$.
	}
	For $\beta = 1$, the same functions are shown in Fig.~\ref{fig:eg-chain-beta1}.
	%
	The candidates are determined by the solid lines,
	e.g., for $\beta = 0$, we have $C_{\alpha,t}$ as
%	e.g., when $\beta = 0$,
%	for $\alpha \geq 2$ all the candidates are singletons,
%	for $\alpha\in [1.5,2)$ the candidates of $0$ and $3$ are singletons and the candidates of
%		$1$ and $2$ are respectively $\Set{0,1}$ and $\Set{2,3}$, etc.
	\begin{align}
		\label{eq:eg-chain-candidates}
%		C_{\alpha, t} = \!\!\!
		\begin{array}{ccccl}
			%C_{\alpha, 0,0} & C_{\alpha, 0,1} & C_{\alpha,0,2} & C_{\alpha,0,3} &
			 t=0 & t=1 & t=2 & t=3 &
			\\
			 \Set{0} & \Set{1} & \Set{2} & \Set{3}, & `a \geq  2 \\
			 \Set{0} & \Set{0,1} & \Set{2,3} & \Set{3}, &  \alpha \in  [1.5,2) \\
			 \Set{0} & V & V & \Set{3}, &  \alpha \in  [\frac{2}{3},1.5) \\
			 V & V & V & V, &  \alpha < 2/3 \ . \\
					%\emptyset, & `a \geq  2/3 
					%\\
					%\Set{0,1,2,3}, & `a < 2/3
		\end{array}
	\end{align}
	%
	Note that each entry in \eqref{eq:eg-chain-candidates} is the inclusion-wise minimum solution to
	\eqref{eq:submodular}. 
	Propositions~\ref{prop:communities-from-candidates} asserts that, for a given $\beta$, the set of
	communities is specified by the blue line as 
	\begin{align}
		\label{eq:eg-chain}
		\begin{array}{c@{\quad}c}
			 %(`a,0)\text{-communities} & (`a,1)\text{-communities} \\[\smallskipamount]
			 \beta = 0 & \beta = 1 \\%[\smallskipamount]
			\!\!
			\!\!
			\left\{
				\!\!\!
				\begin{array}{l@{\ }l}
					\emptyset, & `a \geq  2/3 
					\\
					\Set{0,1,2,3}, & `a < 2/3
				\end{array}\!\!,
			\right.
			&
			\left\{
				\!\!\!
				\begin{array}{l@{\ }l}
					\emptyset, & `a \geq 6
					\\
					\Set{1,2}, & `a \in  [4, 6)
					\\
					\Set{0,1,2,3}, & `a < 4
				\end{array}\!\!.
			\right.
		\end{array}
	\end{align}
	All the communities at $\beta = 0$ (trivial in this example) are web-communities as
	dictated by Corollary~\ref{cor:alpha-beta-supporting}.
	Note also that
	the web-community $\Set{1,2}$ is captured at $`b = 1$ but not $\beta=0$, i.e., when the cost
	function \eqref{eq:fb} rewards the internal influence.%
%	\footnote{A community at $\beta > 0$ is not necessarily a webcommunity in general, although it
%		happened to be the case in this simple example. See Corollary~\ref{cor:alpha-beta-supporting}
%		for the exact condition.}
	This situation may persist in general, and shows the benefit of choosing $\beta > 0$.  Namely, if
	there is a dense subgraph that is moderately connected with the rest of the graph, then it is
	reasonable to favor such a subgraph over a sparse one, even if the sparse subgraph exhibits
	weaker connection to the rest of the graph. 
	%since a dense subgraph may exhibit strong
	%external connections with the rest of the graph compared to the external connections of a
	%sparser subgraph. In this situation, it is intuitive to regard the dense subgraph, rather than
	%the sparse one, as a community as long as the external connections of this dense subgraph are
	%weak compared to its internal ones.
\end{example}
In the above example, the non-trivial community at $\beta =1$ is a web-community, which is
desirable. However, for the same reason mentioned in the example, a community at $\beta > 0$ may
still be desirable even if it is not a web-community. For instance, consider the graph in
Fig.~\ref{fig:eg-star}. The set $\Set{0,1}$ is not a web-community since node~$1$ is more
connected with the rest of the graph than it is with node~$0$. Despite this, it is still desirable
to consider this set as a community since it is the desnsest $2$-subgraph. This set is indeed a
community according to our fomrulation and can be returned at $\beta=1$ and $\alpha\in [2,6)$. 
%See Corollary~\ref{cor:alpha-beta-supporting}

%%%%%%%%%%%

%%%%%%%%%%% fig: plots of f_beta, g_t, and g for chain graph
\begin{figure}
	\centering
	%%%%%%%%%%%%%%%
	\center
	\def\figsep{.1cm}
	\def\dist{1}
	\def\disty{.5}
	%%%%%%%%%%%%%%%
	\subcaptionbox{$`b = 0$ \label{fig:eg-chain-beta0}}{
	\scalebox{0.69}{
		{\def\u{1.3}
			\tikzstyle{point}=[draw,circle,minimum size=.2em,inner sep=0, outer sep=.2em]
			\begin{tikzpicture}[scale=1, x=1.4em,y=1.4em,>=latex]
			\draw[->] (0,-.5*\u) -- (0,8*\u) node [] {};
			\draw[->] (-.5*\u,0) -- (4*\u,0) node [label=right:$`a$] {};
			%ticks
			\foreach \x in {0,...,3}
				\draw (\x*\u,1pt) -- (\x*\u,-3pt);
			\foreach \y in {0,...,7}
				\draw (1pt,\y*\u) -- (-3pt,\y*\u);
			\foreach \xa/\ya/\xb/\yb/\lp/\lb in {
				0/5/3/8/left/{$\kern1em \begin{array}{c} \Set{1} \\ \Set{2} \end{array} \!\!\!, \ `a+5$},
				%0/5/3/8/left/{$\kern1em \Set{1},\Set{2}, `a+5$},
				0/3/2.5/8/left/{$\kern1em \begin{array}{c} \Set{0,1} \\ \Set{2,3} \end{array} \!\!\!, \ 2`a+3$},
				%0/3/2.5/8/left/{$\kern1em \Set{0,1},\Set{2,3}, 2`a+3$},
				0/2/3/5/left/{$\kern1em \begin{array}{c} \Set{0} \\ \Set{3} \end{array} \!\!\!,  `a+2$},
				%0/2/3/5/left/{$\kern1em \Set{0},\Set{3}, `a+2$},
				0/0/2/8/above left/{$\kern1em \Set{0,1,2,3}, 4`a$}
			}
			\draw[dashed] (\xa*\u,\ya*\u) node [inner sep=0,outer sep=0,label={[label distance=0em]\lp:{\scriptsize\lb}}] {} -- (\xb*\u,\yb*\u);
			\path (2/3*\u,8/3*\u) node (c1) [point,red,thick,label=right:{\scriptsize$(2/3,8/3)$}] {};
			\path (1.5*\u,6*\u) node (c2) [point,black,thick,label=right:{\scriptsize$(1.5,6)$}] {};
			\path (2*\u,7*\u) node (c3) [point,black,thick,label=right:{\scriptsize$(2,7)$}] {};
			\draw[-,thick,black] (0,0)--(c1)--(c2)--(c3)--(3*\u,8*\u);
			\draw[-,thick,blue] (0,0)--(c1)--(3*\u,5*\u);
			%
			%\draw[-,thick,blue] (0,0)--(c2)--(c3)--(3*\u,8*\u);
			\path (c3) ++(25:3) node (g1) [] {$\hat{f}_1, \hat{f}_{2}$};
			\path (c1) ++(25:3.5) node (g0) [] {$\hat{f}, \hat{f}_0, \hat{f}_{3}$};
			\end{tikzpicture}}
	}
	}
	%\hfil
	\hspace{-.2cm}
	\subcaptionbox{$`b = 1$ \label{fig:eg-chain-beta1}}{
	\scalebox{0.69}{
		{\def\u{.52}
			\tikzstyle{point}=[draw,circle,minimum size=.2em,inner sep=0, outer sep=.2em]
			\begin{tikzpicture}[scale=1, x=1.4em,y=1.4em,>=latex]
			%\draw[help lines, color=gray!30, dashed] (-.5*\u,-7.5*\u) grid (4*\u,4*\u);
			\draw[->] (0,-14.5*\u) -- (0,7*\u) node [] {};
			\draw[->] (-.5*\u,0) -- (7*\u,0) node [label=right:$`a$] {};
			%%%ticks
			\foreach \x in {0,...,6}
				\draw (\x*\u,1pt) -- (\x*\u,-2pt);
			\foreach \y in {-14,...,6}
				\draw (1pt,\y*\u) -- (-2pt,\y*\u);
			%%%%%%
			\foreach \xa/\ya/\xb/\yb/\lp/\lb in {
				0/0/7/7/left/{$\kern1em \begin{array}{c} \Set{0} \\ \Set{1} \\ \Set{2} \\ \Set{3} \end{array} \!\!\!, \ `a$},
				0/-6/{13/2}/7/left/{$\kern1em \Set{1,2}, 2`a-6$},
				0/-10/{17/3}/7/left/{$\kern1em \begin{array}{c} \Set{0,1,2} \\ \Set{1,2,3} \end{array} \!\!\!, \ 3`a-10$},
				0/-14/{21/4}/7/left/{$\kern1em \Set{0,1,2,3}, 4`a-14$}
			}
			\draw[dashed] (\xa*\u,\ya*\u) node [inner sep=0,outer sep=0,label={[label distance=0em]\lp:{\scriptsize\lb}}] {} -- (\xb*\u,\yb*\u);
			\path (4*\u,2*\u) node (c1) [point,red,thick,label=right:{\scriptsize$(4,2)$}] {};
			\path (5*\u,5*\u) node (c2) [point,black,thick,label=right:{\scriptsize$(5,5)$}] {};
			\path (6*\u,6*\u) node (c3) [point,red,thick,label=right:{\scriptsize$(6,6)$}] {};
			\draw[-,thick,black] (0,-14*\u)--(c1)--(c2)--(c3)--(7*\u,7*\u);
			\draw[-,thick,blue] (0,-14*\u)--(c1)--(c3)--(7*\u,7*\u);
			%%
			\path(c1) ++(-83:4) node(g1){$\hat{f}, \hat{f}_1, \hat{f}_2$};
			\path(c2) ++(160:1.2) node(g0){$\hat{f}_0, \hat{f}_3$};
			\end{tikzpicture}}
	}
	}
	\caption{The plots of $\hat{f}_t(`a)$ (solid) against $`a$ for the undirected graph in
	Fig.~\ref{fig:eg-chain}. The optimal $\hat{f}_t(`a)$ define $\hat{f}(`a)$ (solid blue).
	%, i.e., $T^{*}_{\alpha,0} = \Set{0,3}$ and $T^{*}_{\alpha,1} = \Set{1,2}$.
	%for Example~\ref{eg:chain}
	}
	\label{fig:P}
\end{figure}
%%%%%%%%%%%%%%%%%%%%%%%%%%%%%%%%%%


%%%%%%%%%%%%%%%%%%%%%% end: candidates definition, relation to communities, and example

\subsection{Using maxflow algorithm}

Since $f(C)$ is a submodular function, the minimization problem \eqref{eq:alpha-beta-community} can be solved using any submodular function
minimization (SFM) algorithm. However, a generic SFM algorithm is computationally expensive. Below we reduce the problem of finding the $(\alpha,\beta)$-candidate of any $t\in V$ to the  
mincut problem of the following augmented graph.
\begin{definition}
	\label{def:augmented}
	%Let $(V, A)$ be a digraph and let $d_i:= w(V\backslash \{i\},i)$ be the in-degree of node $i$ for any $i\in V$.
	Let $\alpha, \beta$, and $t$ be as in Definition~\ref{def:alpha-beta} and let $s \notin V$ be
	some additional node.
	%Furthermore, for any $i \in V$, let $d_i:= w(V\backslash \{i\},i)$ be the in-degree of node $i$.
	The
	$(\alpha,\beta,t)$-augmented digraph is the digraph on $V \cup \{s\}$  whose edge weight $w_{\alpha,\beta,t}: (V\cup \{s\})^{2} \rightarrow
	\mathbb{R}_{\geq 0}$ is defined as
	\begin{align}
		w_{`a`b t}(i,j):= \left\{
			\begin{array}{ll}
				w(i,j), & i\in V, j\in V\backslash \{t\} \\
				w(i,j) + \beta d_{i}, & i\in V, j = t  \\
				\alpha, & i = s, j \in V \\
				0, & \text{otherwise}.
			\end{array}
			\right .
	\end{align}
\end{definition}

\begin{theorem}
  \label{thm:mc}
	The candidate $C_{`a,t}$ is the unique inclusion-wise minimum set $C$ such that $(\{s\}\cup
	V\backslash C, C)$ is an $s$--$t$ mincut of the $(\alpha,\beta,t)$-augmented digraph.
\end{theorem}

\begin{Proof}[Sketch]
  The theorem follows by showing that for any $C\subseteq V:t\in C$, the incut function of the
  augmented graph evaluated at $C$ is equal to
  $f(C) + \alpha \! \cdot \! |C|$ plus some constant independent of $C$, i.e., \eqref{eq:alpha-beta-community} is
  equivalent to the $s$--$t$ mincut
  problem on the augmented graph.
\end{Proof}

% Due to this theorem, in the subsequent discussions, we will interchangeably refer to the
% $(\alpha,\beta)$-community of $t$ as the
% minimum minimizer of \eqref{eq:alpha-beta-community} or the minimum $s$--$t$ mincut of the augmented
% digraph.

For a given $`a$ and $`b$, the set $\mcC$ of communities can be computed as follows:
\begin{enumerate}
\item Run the maxflow algorithm $n$ times, one for each $t\in V$ on the $(`a,`b,t)$-augmented graph to obtain $\hat{f}_t(`a)$ and its unique minimum solution $C_{`a,t}$.
\item Compute $\hat{f}(`a)$ as the minimum $\hat{f}_t(`a)$ over $t\in V$, and retain only the associated non-singleton $C_{`a, t}$ for $\mcS_{`a}$. 
\end{enumerate}
The second step can run in linear time and so the first step gives the overall complexity.

We can use the parametric maxflow algorithm of~\cite{gallo89} to compute $\hat{f}_t(`a)$ for all $`a\geq 0$. The running time of the parametric algorithm is indeed the same as that of the push-relabel maxflow algorithm. Hence, we only need $n$ maxflow algorithms on the augmented graphs to obtain $\hat{f}_t(`a)$ and $C_{`a,t}$ for all $`a\geq 0$ and $t\in V$. We can then compute $\hat{f}(`a)$ as the minimum $\hat{f}_t(`a)$ over $t\in V$, and retain the associated non-singleton $C_{`a,t}$ for $\mcS_{`a}$. This step can be computed in $O(n^2)$ because there are at most $n$ line segments for each $\hat{f}_t(`a)$ and $\hat{f}(`a)$. 

