
\section{Community hierarchy}
\label{sec:hierarchy}

The following result shows that, similar to hierarchical clustering methods, the set of communities
forms a hierarchy and so can be represented by a dendrogram in linear storage.

\begin{theorem}
  \label{thm:hierarchy}
  For $`a_1\geq `a_2\geq 0$ and $C_i\in \mcS_{`a_i}$ for $i=1,2$, we have $C_1\subseteq C_2$ or
  $C_1\cap C_2=`0$. Furthermore, $C_1\subsetneq C_2$ implies $`a_1>`a_2$. Hence, the set $\mcC$ of communities can be represented by a dendrogram with $`s$ measuring the cophenetic similarity of the dendrogram.
\end{theorem}

The hierarchical structure can be observed from Example~\ref{eg:mcC}, where matched pairs $C_i$'s
are disjoint communities with the same strength $`s(C_i)=1$. Since the trivial community $V$
contains a matched pair, it has a strictly smaller strength $`s(V)=0$ as expected.
We remark that the proof of the theorem relies only on the submodularity~\eqref{eq:submodular} of
$f_{\beta}$, and so the theorem extends to submodular functions that are not necessarily defined in
terms of graph cut. 

To understand the parameter $`a$ as a measure of similarity, we will strengthen
Proposition~\ref{pro:single-deviation} to give a more precise interpretation of $`a$ as a bound
on the marginal change in the cost of a community.
\begin{theorem}
  \label{thm:f-shrink-expand}
  Each $C\in \mcS_{`a}$ satisfies%\footnote{The converse of the theorem is not true. For instance, recall in Example~\ref{eg:two-chains} that $\Set{2,3}$ is not an $(`a,`b)$-community for $`a=1.7$ and $`b=1$. However, it can be shown that $\Set{2,3}$ satisfies \eqref{eq:f-shrink} and \eqref{eq:f-expand}.}
  \begin{subequations}
    \label{eq:f-shrink-expand}
    \begin{align}
      \label{eq:f-shrink}
      \alpha &< \min_{B\subsetneq C:|B|\geq 1} \frac{f(B) - f(C)}{|C`/ B|}, \\
      \label{eq:f-expand}
      \alpha &\geq \max_{A\subseteq V:A\supsetneq C} \frac{f(C) - f(A)}{|A`/ C|}. 
    \end{align}
  \end{subequations}
  (By convention, we set the r.h.s. of \eqref{eq:f-expand} to $-\infty$ if $C=V$.)
\end{theorem}

In other words, $`a$ is both a lower bound~\eqref{eq:f-shrink} of the marginal increase in the cost
$f$ when the community shrinks, and an upper bound~\eqref{eq:f-expand} of the marginal decrease in
the cost $f$ when the community expands. The above theorem can be viewed as a generalization of
Proposition~\ref{pro:single-deviation} because \eqref{eq:single-shrink} and \eqref{eq:single-expand}
are the special cases when we further impose $B=C`/\Set{i}$ for $i\in C$ in \eqref{eq:f-shrink}, and
$A=C\cup \Set{i}$ for $i\in V`/C$ in \eqref{eq:f-expand} respectively. As the following corollary
shows, the result also ties back to the notion of graph expansion used to measure cluster quality.

\begin{corollary}
  Each $C\in \mcS_{`a}$ satisfies for $`b=0$
  \begin{subequations}
    \begin{align}
      \label{eq:bound-beta0}
      \frac{w(V\backslash C, C)}{|V\backslash C|} 
      &
        \stackrel{ \rm{(i)} }{\leq}
        \alpha
        \stackrel{ \rm{(ii)} }{<}
        \min_{B\subsetneq C: |B| \geq 1} \frac{w(C\backslash B, B)}{|C\backslash B|} 
    \end{align}
    and, for $`b=1$,
    \begin{align}
      % 
      \label{eq:bound-beta1}
      \max_{A\subseteq V: A\supsetneq C} \frac{w(A,A)-w(C,C)}{|A\backslash C|}  
      &
        \stackrel{ \rm{(iii)} }{\leq}
        \alpha
        \stackrel{ \rm{(iv)} }{<}
        \frac{w(C, C)}{|C|-1}. 
    \end{align}
  \end{subequations}
  ~\relax
\end{corollary}
\begin{Proof}
  (ii) and (iii) follow from \eqref{eq:f-shrink} and \eqref{eq:f-expand} with $\beta=0$ and $1$,
  respectively. (i) and (iv) follow from \eqref{eq:f-shrink-expand} by choosing
	$C' = V$ 
	and
	$C'\subsetneq C:|C'|=1$
	with $\beta = 0$ and $1$, respectively.
\end{Proof}

